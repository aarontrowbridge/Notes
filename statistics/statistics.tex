\documentclass{article}

\usepackage[bookmarks]{hyperref}
\usepackage{amsmath}
\usepackage{enumitem}

\newtheorem{mydef}{Definition}

% \usepackage{pst-poker}
% \psset{linewidth=25pt}

\linespread{1.2}


\title{Notes: Statistics}
\author{Aaron Trowbridge}
\date{}

\setcounter{section}{-1}

\begin{document}
\maketitle

\pagenumbering{roman}
\tableofcontents
\newpage


\pagenumbering{arabic}


\section{Introduction}

In the beginning there was a population, and over that population, there was a distribution.  As mortals, and not gods, if we want to try to understand this distribution the best we can do is attempt to observe features of the distribution--hopefully at random--and use that data to build a blurry picture of the truth lying behind our perception.  

Dramatization aside, this is the principle assumption and goal of statistics: infer knowledge about a hidden (parameterized) distribution by taking random samples and applying mathematical techniques to the data obtained. Usually we are either trying to estimate the \textit{true} hidden parameters or test hypotheses we might have about those parameters. 

In these notes I intend to condense the essential methods needed to piece together an actionable representation of the uncertain world around us.  Statistical analysis of data is an essential tool in almost every field imaginable: from physics to psychology to finance, and everywhere in between.

Throughout these notes I will include custom visualizations and code snippets, for which I will be using the Julia programming language.  Code for everything in this document, including the \LaTeX \  source code, can be found on my github. 

\newpage
\part{Data: Production, Visualization, and Description}
\section{Producing Data}

There are many ways to get data, not all are equal, and, even more importantly, not all are equally as useful. \textit{Bias} can seep into data in a number of ways depending on how the data is obtained. 

The first step in getting data is to decide on the \textit{population} we are interested in, which is typically very large (if it weren't we could just sample the whole population and there would be no need for statistics!). 

The next step is to decide what kind of data we are interested in: namely either \textit{numerical} (e.g. height or weight) or \textit{categorical} (e.g. political party or favorite flavor of ice cream). Denoting our \textit{random} variable of interest as $X$: 

\begin{mydef}[Numerical Variable] 
A feature obtained from an observation which has a random numerical value according to the underlying distribution, e.g. $X \sim N(\mu, \sigma)$.
\end{mydef}

\begin{mydef}[Categorical Variable] 
A feature obtained from an observation which takes it's value from a set of possible categories $\{x_1, x_2, \dots , x_n\}$ and $P(X = x_i) = p_i$
\end{mydef}

The final step is to decide how we plan on getting our data; procedures for which can be broken up into two categories: \textit{sampling} and \textit{experimenting} 

\subsection{Sampling}

Since populations are so large, and we can't just issue a census, we need a way to get a representative \textit{sample} (say of size $n$) of the population, that we can analyze.  There are various ways to sample and each comes with it's advantages and disadvantages.
\bigbreak
\noindent Here I will list some of the typical sampling methods:

\begin{description}[style=nextline]
\item [Simple Random Sample] This is the ideal sample and involves enumerating all of the possible members of the population and randomly selecting $n$ numbers in the range of the population size

\item [Survey Sample] Here $n$ 

\item [Stratified Random Sample]
    
\end{description}

\subsection{Experimenting}


\section{Data Visualization}
\section{Descriptive Statistics}

\newpage
\part{Probability Theory}
\section{Probability}
\section{Random Variables}
\section{Expectation}
\section{Important Inequalities}
\section{Convergence Results}
\section{Distributions}
\subsection{Normal Distribution}
\subsection{$t$-Distribution}
\subsection{$\chi^2$-Distribution}

\newpage
\part{Statistical Inference}
\section{Parameter Estimation}
\section{Hypothesis Testing}

\newpage
\part{Statistical Models and Methods}
\section{Regression}
\section{Classification}

    
\end{document}